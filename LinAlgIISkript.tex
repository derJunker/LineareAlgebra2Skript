\documentclass {article}

\usepackage[a4paper, total={6in, 8in}]{geometry}

\usepackage{amssymb}
\usepackage{amsfonts}
\usepackage{enumitem}
\usepackage{stmaryrd}

%-------------------------------commands definitionen-------------------------------
%für methoden die eine fallunterscheidung haben aufrufen mit: 
%\twopartdef { x } {x \geq 0} {-x} {x < 0} (bsp. Betrag)
\newcommand{\twopartdef}[4] {
	\left\{
		\begin{array}{ll}
			#1 & #2 \\
			#3 & #4
		\end{array}
	\right.
}
%für methoden mit 3 fällen zwischen denen entschieden werden muss
\newcommand{\threepartdef}[6]
{
	\left\{
		\begin{array}{lll}
			#1 & \mbox{if } #2 \\
			#3 & \mbox{if } #4 \\
			#5 & \mbox{if } #6
		\end{array}
	\right.
}

%abkürzung von \textnormal
\newcommand{\tn}[1]{\textnormal {#1}}
%-----------------------------ende commands definitionen-----------------------------

\begin{document}
\title{Lineare Algebra II - Mitschrift}
\author{Sebastian Pretzsch, Finn Ribbeck, Jonas Heitz}
\maketitle
%------------------------------------Vorlesung 1------------------------------------
\section{Relationen}
Relationen beschreiben Beziehungen zwischen Elementen von Mengen. \\
Wdh. $X{\times}Y := \{(x,y) | x \in X, y \in Y\}$ \\
Menge der (geordneten) Paare, "kartesiches Produkt".

\subsection*{Definition 1.1.}
Seien X,Y Mengen. Eine \textbf{Relation} zwischen X und Y ist eine Teilmenge $R \subseteq X{\times}Y$. \\
\underline{Notation}: für $(x,y)\in R$ auch $xRy$. \\
Falls $X=Y$ sage auch Relation "auf X".

\subsection*{Beispiel}
$X$ Punktmenge, $Y$ Geradenmenge
$$xRy :\Leftrightarrow \tn{Punkt x liegt auf Gerade y}$$

\subsection*{Bemerkung}
Eine Abbildung (Funktion) $f:X{\to}Y$ weist zu jedem $x\in X$ genau ein $
y\in Y$ zu. \\
Wir werden diese fortan als spezielle Relation
$$R_f = \{(x, f(x)) | x\in X\} \subseteq X{\times}Y$$
auffassen. Wir betrachten insbesondere Relation auf X.

\subsection*{Definition 1.2.}
Sei X eine Menge. Eine Relation $R \subseteq X{\times}X$ ist
\begin{enumerate}[label=(\alph*)]
\item \textbf{reflexiv} falls $xRx \hspace*{2mm}\forall x\in X$,
\item \textbf{symmetrisch} falls $xRy \Rightarrow yRx \hspace*{2mm}\forall x,y\in X$ (denn auch "$\Leftarrow$" gilt),
\item \textbf{anti-symmetrisch} falls $xRy \land yRx \Rightarrow x=y$,
\item \textbf{transitiv} falls $xRy \land yRz \Rightarrow xRz$.
\end{enumerate}

\subsection*{Beispiel} 
Sei $X = \mathbb R$.
\begin{enumerate}[label=(\alph*)]
\item $ R:=\{(x,x) | x\in X\}$ (d.h. $xRy \Rightarrow x=y$) erfüllt a), b), c), d).
\item $ xRy :\Leftrightarrow |x|=|y|$ erfüllt a), b), d) (nicht c), da $|1|=|-1|$)
\item $ xRy :\Leftrightarrow x\leq y$ erfüllt a), c), d) (nicht b), da $2\leq 3$ aber $2 \neq 3$
\end{enumerate}
Nun definieren wir die wichtigsten Arten von Relationen.

\subsection*{Definition 1.3.}
Eine Relation R auf X heißt
\begin{enumerate}[label=(\arabic*)]
\item \textbf{Äquivalenzrelation} falls sie reflexiv, symmetrisch und transitiv ist (typische Notation "~"),
\item \textbf{Ordnungsrelation} falls sie reflexiv, anti-symmetrisch und transitiv ist (Notation "$\leq$").
\end{enumerate}
Im Fall 2. heißt R auch \textbf{(partielle) Ordnung}, und falls 
$$ xRy \lor yRx \hspace*{2mm} \forall x,y \in X$$
zusätzlich gilt \textbf{totatle/ lineare Ordnung}.\\
\textbf{Äquivalenzrelationen} auf X entsprechen genau den Zerlegungen von X.
Zwei Mengen A, B heißen \textbf{disjunkt} falls $A \cap B = \emptyset$.

\subsection*{Beispiel}
Kleiner Ausblick:
\begin{enumerate}[label=(\alph*)]
\item Zwei Matrizen $A,B \in \tn{Mat}_{n{\times}m}(K)$ heißen \textbf{ähnlich} falls $S \in \tn{GL}_n(K)$ existiert mit \\$B=S^{-1}{\cdot}A{\cdot}S$.
\item Sei M Menge und $X:=\mathcal P(M)$ Potenzmenge von M\\
 Zu $A,B \in X$ definiere $A\leq B :\Leftrightarrow A\subseteq B$ eine Ordnungsrelation.
\end{enumerate}

%------------------------------------Vorlesung 2------------------------------------
\subsection*{Definition 1.4.}
Sei X Menge. Eine \textbf{Zerlegung} von X ist eine Menge $\mathcal{C}$ von paarweise disjunkten nicht-leeren Mengen $A \subseteq X$ mit $\bigcup_{A \in \mathcal{C}} A = X$.

\subsection*{Beispiel}
Eine Zerlegung von $\{ a, b, c\}$ ist $\{\{ a\}, \{ b, c\}\}$.

\subsection*{Lemma 1.5.}
Sei $\sim$ Äquivalenzrelation auf X und zu $x \in X$ sei $[x] := \{y \in X | x \sim y\} \subseteq X$ ("Klasse" von X).\newline
Dann ist $\mathcal{C} := \{[x] | x \in X\}$ eine Zerlegung von X.\newline
Ferner gilt $x \sim y \Leftrightarrow [x] =  [y]$. (*)\newline
\subsubsection*{Beweis (*)}
"$\Leftarrow$": $y \in [y] = [x] \Rightarrow x \sim y$.\newline
"$\Rightarrow$": Gelte $x \sim y$. Zeige [x] = [y].
\begin{itemize}
\item[--] "$\subseteq$": $z \in [x] \Rightarrow x \sim z \Rightarrow y \sim z \Rightarrow z \in [y]$
\item[--] "$\supseteq$": $z \in [y] \Rightarrow y \sim z \Rightarrow x \sim z \Rightarrow z \in [x]$\newline
\end{itemize}
Zeige $\mathcal{C}$ Zerlegung.\newline
Angenommen $[x] \cap [y] \neq \emptyset \Rightarrow \exists z \in X: z \in [x] \cap [y] \Rightarrow \exists z \in X: x \sim z \sim y \Rightarrow x \sim y \Rightarrow [x] = [y].$\newline Also sind die Klassen paarweise disjunkt (und nicht-leer, da $\forall x \in X: x \in [x]$).\newline Wegen $\forall x \in X: x \in [x]$ gilt außerdem $\bigcup_{x \in X} [x] = X$.

\begin{flushleft}
$\square$\\
\end{flushleft}

\subsection*{Beispiel}
\begin{enumerate}[label=(\alph*)]
\item $X = \mathbb{R}, x \sim y :\Leftrightarrow |x| = |y|$.
Zerlegung von $\mathbb{R}$ ist $\{\{0\}\} \cup \{\{a, -a\} | a \in \mathbb{R}_{>0}\}$.
\item $X = \mathbb{Z}, n \in \mathbb{N}_{>0}$, betrachte $x \sim y :\Leftrightarrow n | x - y \Leftrightarrow y = x + k \cdot n$ für ein $k \in \mathbb{Z}$.
Zerlegung von $\mathbb{Z}$ ist $\{0 + n\mathbb{Z},\ldots, (n - 1) + n\mathbb{Z}\}$ (n Klassen).
\end{enumerate}

\subsection*{Definition 1.6.}
Die Zerlegung von X bezüglich der Relation $\sim$ heißt Zerlegung in \textbf{Äquivalenzklassen}. Notation für $\mathcal{C}$ auch X/$\sim$, "X modulo $\sim$".
\subsubsection*{Bemerkung}
Umgekehrt definiert jede Zerlegung $\mathcal{C}$ von X Äquivalenzrelation $\sim_\mathcal{C} := \bigcup_{A \in \mathcal{C}} A \times A \subseteq X \times X$ (Übung) und die Konstruktionen sind zueinander invers.

\subsection*{Beispiel}
Es gibt fünf Zerlegungen von $X = \{a, b, c\}$. Die Kreuztabellen der zugehörigen Äquivalenzrelationen lauten:
\begin{tabbing}
\=
\begin{tabular}[h]{cccc}
 & a & b & c \\
a & X & & \\
b & & X & \\
c & & & X \\
\end{tabular}
\=
\begin{tabular}[h]{cccc}
 & a & b & c \\
a & X & X & \\
b & X & X & \\
c & & & X \\
\end{tabular}
\=
\begin{tabular}[h]{cccc}
 & a & b & c \\
a & X & & X \\
b & & X & \\
c & X & & X \\
\end{tabular}
\=
\begin{tabular}[h]{cccc}
 & a & b & c \\
a & X & & \\
b & & X & X \\
c & & X & X \\
\end{tabular}
\=
\begin{tabular}[h]{cccc}
 & a & b & c \\
a & X & X & X \\
b & X & X & X \\
c & X & X & X \\
\end{tabular}
\\
\>$\{\{a\}, \{b\}, \{c\}\}$
\>$\{\{a, b\}, \{c\}\}$
\>$\{\{a, c\}, \{b\}\}$
\>$\{\{b, c\}, \{a\}\}$
\>$\{\{a, b, c\}\}$\\
\end{tabbing}
Zusammenhang mit Abbildungen:
\subsubsection*{Bemerkung}
Jede Äquivalenzrelation $\sim$ auf X definiert eine (surjektive) Abbildung\\
$\pi: X \longrightarrow X/\sim, x \longmapsto [x]$.\\
Umgekehrt definiert jede Abbildung $f: X \longrightarrow Y$ eine Äquivalenzrelation auf X durch\\ $x \sim x' :\Leftrightarrow f(x) = f(x')$.

\subsection*{Wichtige BegrLeftrightarrowe zu Ordnung}
Sei (X, $\leq$) geordnete Menge. Darstellung (im endlichen Fall) als Diagramm mit Verbindungen \begin{tabular}{c}
b \cr $\mid$ \cr a
\end{tabular} falls $a < b$ und es existiert kein $c$ mit $a < c < b$ "Nachbarschaftsrelation".

\subsection*{Beispiel}
siehe Vorlesung (15.04.21 / 56:20)

\subsection*{Definition 1.7.}
Sei (X, $\leq$) geordnete Menge. Elemente $a, b \in X$ heißen \textbf{vergleichbar} falls $a \leq b \lor b \leq a$.\\
Teilmenge $Y \subseteq X$ heißt \textbf{Kette} falls alle $a, b \in Y$ vergleichbar (d.h. X ist Kette gdw. X total geordnet).
\begin{tabbing}
Ein Element $a \in X$ heißt \= \textbf{maximal} falls $\forall x \in X: x \geq a \Rightarrow x = a$,\\ \>\textbf{größtes Element} falls $\forall x \in X: x \leq a$.\\ \>\textbf{obere Schranke} von $Y \subseteq X$ falls $\forall y \in Y: y \leq a$.\\
Dual dazu: minimal, kleinstes Element, untere Schranke\\
\end{tabbing}

\subsection*{Beispiel}
$\mathcal{P}(M)\backslash\{\emptyset\}$ hat als minimale Elemente $\{m\}$ wobei $m \in M$.\\
z.B.: $M:=\{a, b\} \rightarrow \mathcal{P}(M)\backslash\{\emptyset\} = \{\{a\}, \{b\}, \{a, b\}\}$ hat minimale Elemente $\{a\}$ und $\{b\}$.
%----------------------------------Vorlesung 3----------------------------------
\section{Unendliche Mengen und Lemma von Zorn}
Notation: Zu $n\in\mathbb{N}$ sei $[n]:= \{1,2...,n\} $.

\subsection*{Definition 2.1}
Eine Menge $X$ heißt endlich, falls $\exists n \in \mathbb{N}$ und Bijektion $\phi: X \rightarrow [n]$; in diesem Fall heißt $n$ die Kardinalität $|X|$ oder $\#X$ von $X$.
Ansonsten heißt $X$ unendlich. 
\subsubsection*{Bemerkung (Taubenschlagprinzip)}
Ist $n>m$, so ist jede Abbildung $f: [n] \rightarrow [m]$ nicht injektiv, das heißt $\exists x\neq x'$ ("Tauben") mit $f(x) = f(x')$. \\
Es folgt: Ist $\phi : [n] \rightarrow [m]$ bijektiv, so gilt $n=m$. Somit ist die Kardinalität eindeutig.

\subsection*{Beispiel}
Menge $X = \{a,b,c,d\}$ ist endlich mit Kardinalität 4, \\
$\#X =  4$ mögliche Bijektion: 
\begin{tabular}{cccc}
$a \rightarrow 1$ & $c \rightarrow 3$ \\
$b \rightarrow 2$ & $d \rightarrow 4$ \\
\end{tabular}
\\
\\
Der Umgang mit unendlichen Mengen erfordert oft "Auswahlaxiom".

\subsection*{Bemerkung}
Abbildung $f: X \rightarrow Y$ ist injektiv $\Leftrightarrow \exists g: Y \rightarrow X : g \circ f = id_x$ ($X \neq \emptyset $).
\subsubsection*{Beweis}
$"\Leftarrow"$ \\
$x,x' \in X :$
$f(x) = f(x') \Rightarrow  x = g(f(x)) = g(f(x'))$ nach Voraussetzung.\\
Also ist f injektiv. \\ 
\\
$"\Rightarrow"$\\
Sei $Y_0 := im f$ und betrachte $f_0 : X \rightarrow Y_0$, $f_0(x) := f(x)$ ist bijektiv \\
 $\Rightarrow \exists g_0 := f_0^{-1} Y_0 \rightarrow X$,\\
setze fort zu $g: Y \rightarrow X$,\\
das heißt $g(y) := g_0(y)$ für $y\in Y_0$, sonst beliebig.\\
$\Rightarrow g(f(x)) = x \forall  x \in X (f(x) \in Y_0)$.\\
\begin{flushright}
$\square$\\
\end{flushright}
Gilt $f \circ g = id_y$, so ist $f$ surjektiv: zu $y\in Y \exists x:= g(y)$ mit $f(x) = y$.\\
Umgekehrt gilt:
\subsection*{Axiom 2.2 (Auswahlaxiom)}
Jede surjektive Abbildung $f: X \rightarrow Y$ besitzt eine Rechtsinverse, das heißt \\
$g: Y \rightarrow X$ mit $f \circ g = id_y$.\\
Äquivalente Formulierung:\\
Sei $(A_i)_{i \in I}$ Familie von nicht-leeren Teilmengen von X.\\
Dann existiert $(a_i)_{i \in I}$ mit $a_i \in A_i \forall i \in I$.\\
(Kurz: $X_{i \in I} A_i \neq \emptyset $)
\subsubsection*{"Beweis"}
Jedes $y \in Y$ hat ein Urbild $x \in X$ mit $f(x) = y$.\\
Definiere also $g: Y \rightarrow X$ durch $Y \rightarrow X$ mit $f(x) = y$.\\
Dass eine solche Auswahl stets existiert, besagt das Axiom.

\subsection*{Definition 2.3}
Zwei Mengen $X,Y$ heißen gleichmächtig, falls eine Bijektion $\phi X \rightarrow Y$ existiert; $X$ hat Mächtigkeit $Y$.(Dies erfüllt Eigenschaften einer Äquivalenzrelation)\\
Eine Menge $X \neq \emptyset$ heißt abzählbar, falls Surjektion $\mathbb{N} \rightarrow X$ existiert. ($\emptyset$ sei abzählbar; $f: \mathbb{N} \rightarrow X$ surjektiv $\Rightarrow X = \{f(0), f(1), ...\}$), das heißt (Auswahlaxiom) falls  Injektion $X \rightarrow \mathbb{N}$ besteht.\\
Andernfalls heißt $X$ überabzählbar.

\subsection*{Bemerkung}
\begin{enumerate}[label=(\arabic*)]
\item abzählbar unendlich  $\Leftrightarrow$ Mächtigkeit $\mathbb{N}$
\item existiert $f: X \rightarrow \mathbb{N}$ mit endlichen Urbildern $f^{-1}(\{n\}) \forall n\in \mathbb{N}$, so ist $X$ abzählbar.
\end{enumerate}

\subsection*{Satz 2.4}
$\mathbb{Q}$ ist abzählbar.
\subsubsection*{Beweis (1. Cantorscher Diagonalbeweis)}
siehe Vorlesung.

\subsection*{Satz 2.5}
\begin{enumerate}[label=(\alph*)]
\item $\mathbb{R}$ ist überabzählbar.
\item Für jede Menge $X$ ist $P(X)$ nicht gleichmächtig zu $X$.
\end{enumerate}
\subsubsection*{Beweis (2. Cantorscher Diagonalbeweis)}
siehe Vorlesung.
%------------------------------------Vorlesung 4------------------------------------
\subsection*{Satz 2.6. (Lemma von Zorn)}
Sei $(X, \leq)$ geordnete Menge sodass jede Kette in $Y \subseteq X$ eine obere Schranke hat. Dann hat $X$ ein maximales Element.\\
\subsubsection*{Diskussion:}
\begin{enumerate}[label=(\arabic*)]
\item Nicht interessant (trivial) für $X$ endlich, oder für $(X,\leq )$ total geordnet: betrachte Kette $Y=X$.
\item Anwendungsfall oft $X\subseteq \mathcal P(M)$. Kette $Y\subseteq X$ hat obere Schranke z.B. 
falls $\bigcup_{A\in Y}A\in X$ gilt.
\item Beweis ist technisch anspruchsvoller (Mengenlehre) und benutzt Auswahlaxiom; umgekehrt kann Auswahlaxiom aus Zorn folgern (vgl. Halmos).
\end{enumerate}
Nun zu Basen für beliebige Vektorräume. \\
In Lineare Algebra I haben wir gesehen, dass jeder endlich erzeugte K-Vektorraum V eine (endliche) Basis $S\subseteq V$ hat; \\
Dann gilt $V\cong K^n$, $n:=\tn{dim}_KV$.\\
\underline{Erinnerung:} Eine Teilmenge $S\subseteq V$ heißt Erzeugendensystem falls 
$$span \hspace*{1mm} S := \{\sum_{i=1}^n
\lambda_i v_i | \lambda_i \in K, v_i \in S, n \in \mathbb N\} = V \tn{ gilt.}$$
Und $S\subseteq V$ heißt linear unabhängig falls $\sum_{i=1}^n \lambda_i v_i = 0$ mit $v_i\in S$ verschieden stets $\lambda_i = 0$, $\forall i$ impliziert.
\subsection*{Proposition 2.7.}
Zu I Menge sei $K^{(I)} := \{f:I\rightarrow K | f(i) \neq 0$ nur für endliche viele $i\in I\}$. \\
Dann ist $K^{(I)}$ ein K-Vektorraum mit Basis $\{e_i|i\in I\}$, wobei $e_i(j):= \twopartdef { 1 } {\tn{falls } j=i} {0} {\tn{sonst}}$. \\
Umgekehrt ist jeder K-Vektorraum mit Basis isomorph zu einem $K^{(I)}$.
\subsubsection*{Beweis:}
\begin{itemize}
\item[--] "K-Vektorraum": $K^{(I)}$ ist Untervektorraum von $K^I := \{f:I\rightarrow K\}$.
\item[--] "Erzeugendensystem": sei $f\in K^{(I)}$ und sei $I_0 := \{i_1,...,i_n\} \subseteq I$ mit $f(i) = 0$ $\forall i \notin I_0;$ setze $\lambda_k := f(i_k)$, $k=1,...n$ 
$$ \Rightarrow f = \sum_{k=1}^n \lambda_k e_{i_k}\tn{, denn }\lambda_j = f(i_j) = \sum_{k=1}^n \lambda_k e_{i_k}(i_j) = \lambda_j, \hspace*{1mm} \forall j = 1,...,n$$
\item[--] "linear unabhängig": angenommen $\sum \lambda_k e_{i_k} = 0 \Rightarrow \lambda_j = \sum \lambda_k e_{i_k}(i_j) = 0$ $\forall j$
\end{itemize}
Zusatz: Sei V ein K-Vektorraum mit Basis $\{e_i | i \in I\}$, so ist die Linearkombinationsabbildung (vgl. Lin-Alg.I, Bem. 8.3)
$\gamma: K^{(I)} \rightarrow V$, $f\mapsto \sum_{i\in I} f(i)e_i$ ein K-Isomorphismus.
\begin{flushright}
$\square$\\
\end{flushright}
\subsection*{Bemerkung}
$K^{(I)}=K^I \Leftrightarrow$ I endlich. \\
Aber z.B. ist $\mathbb Q^{(\mathbb N)}$ abzählbar und $\mathbb Q^\mathbb N$ überabzählbar (Übung). \\
Basen z.B. für $K^I$?
\subsection*{Satz 2.8.}
Jeder Vektorraum besitzt eine Basis.
\subsubsection*{Beweis (via Lemma von Zorn)}
Sei V ein K-Vektorraum und sei $X \subseteq \mathcal P(V)$ die Menge aller linear unabhängigen Teilmengen $S \subseteq Y$. \\
Sei $Y\subseteq X$ Kette. Behauptung: $T:=\bigcup_{S\in Y} S$ ist linear unabhängig. \\
Gelte $\sum_{i=1}^n \lambda_i v_i = 0$ mit $\lambda_i \in K$ und $v_i \in T$ verschieden. \\
$$\Rightarrow \exists S_i \in Y\tn{ mit }v_i \in S_i\tn{, }\forall i=1,...n$$
$$\Rightarrow_{\tn(S_1,...,S_n \tn{Kette)}} \exists i_0\tn{ mit }S_i \subseteq S_{i_0}\tn{, } \forall i\tn{, somit }v_i \in S_{i_0}$$
$$\Rightarrow_{S_{i_0}\tn{ lin. unabh.}}\tn{ alle }\lambda_i=0$$
$$\Rightarrow_{\tn{Satz 2.6 (Zorn)}}\tn{ es existiert ein maximales Element S in X.}$$
Behauptung dieses S ist auch erzeugend (dann fertig).\\
Angenommen es existiert $v\in V\backslash\tn{span }S$ sei $S':= S\cup \{v\}$. Dann ist S' linear unabhängig.\\
Betrachte $\sum_{i=1}^n\lambda_iv_i=0$ mit $v_i\in S'$ verschieden.
\begin{itemize}
\item falls alle $v_i\in S \Rightarrow$ alle $\lambda_i = 0$
\item sonst $\lambda v = \sum \lambda_j v_j$, wäre $\lambda \neq 0 \Rightarrow v\in \tn{span }S \lightning$\\
$\Rightarrow \lambda = 0 \Rightarrow$ alle $\lambda_j=0$
\end{itemize}
Weil aber $S\subsetneq S'$ Widerspruch zur Maximalität von S.
\begin{flushright}
$\square$\\
\end{flushright}
%----------------------------------Vorlesung 5----------------------------------
\end{document}