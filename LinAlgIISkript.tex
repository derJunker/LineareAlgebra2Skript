\documentclass {article}

\usepackage[a4paper, total={6in, 8in}]{geometry}

\usepackage{amssymb}
\usepackage{amsfonts}
\usepackage{enumitem}
\usepackage{stmaryrd}

%-------------------------------commands definitionen-------------------------------
%für methoden die eine fallunterscheidung haben aufrufen mit: 
%\twopartdef { x } {x \geq 0} {-x} {x < 0} (bsp. Betrag)
\newcommand{\twopartdef}[4] {
	\left\{
		\begin{array}{ll}
			#1 & \mbox{if } #2 \\
			#3 & \mbox{if } #4
		\end{array}
	\right.
}
%für methoden mit 3 fällen zwischen denen entschieden werden muss
\newcommand{\threepartdef}[6]
{
	\left\{
		\begin{array}{lll}
			#1 & \mbox{if } #2 \\
			#3 & \mbox{if } #4 \\
			#5 & \mbox{if } #6
		\end{array}
	\right.
}

%abkürzung von \textnormal
\newcommand{\tn}[1]{\textnormal {#1}}
%-----------------------------ende commands definitionen-----------------------------

\begin{document}
\title{Lineare Algebra II - Mitschrift}
\author{Sebastian Pretzsch, Finn Ribbeck, Jonas Heitz}
\maketitle
%------------------------------------Vorlesung 1------------------------------------
\section{Relationen}
Relationen beschreiben Beziehungen zwischen Elementen von Mengen. \\
Wdh. $X{\times}Y := \{(x,y) | x \in X, y \in Y\}$ \\
Menge der (geordneten) Paare, "kartesiches Produkt".

\subsection*{Definition 1.1.}
Seien X,Y Mengen. Eine \textbf{Relation} zwischen X und Y ist eine Teilmenge $R \subseteq X{\times}Y$. \\
\underline{Notation}: für $(x,y)\in R$ auch $xRy$. \\
Falls $X=Y$ sage auch Relation "auf X".

\subsection*{Beispiel}
$X$ Punktmenge, $Y$ Geradenmenge
$$xRy :\iff \tn{Punkt x liegt auf Gerade y}$$

\subsection*{Bemerkung}
Eine Abbildung (Funktion) $f:X{\to}Y$ weist zu jedem $x\in X$ genau ein $
y\in Y$ zu. \\
Wir werden diese fortan als spezielle Relation
$$R_f = \{(x, f(x)) | x\in X\} \subseteq X{\times}Y$$
auffassen. Wir betrachten insbesondere Relation auf X.

\subsection*{Definition 1.2.}
Sei X eine Menge. Eine Relation $R \subseteq X{\times}X$ ist
\begin{enumerate}[label=(\alph*)]
\item \textbf{reflexiv} falls $xRx \hspace*{2mm}\forall x\in X$,
\item \textbf{symmetrisch} falls $xRy \Rightarrow yRx \hspace*{2mm}\forall x,y\in X$ (denn auch "$\Leftarrow$" gilt),
\item \textbf{anti-symmetrisch} falls $xRy \land yRx \Rightarrow x=y$,
\item \textbf{transitiv} falls $xRy \land yRz \Rightarrow xRz$.
\end{enumerate}

\subsection*{Beispiel} 
Sei $X = \mathbb R$.
\begin{enumerate}[label=(\alph*)]
\item $ R:=\{(x,x) | x\in X\}$ (d.h. $xRy \Rightarrow x=y$) erfüllt a), b), c), d).
\item $ xRy :\iff |x|=|y|$ erfüllt a), b), d) (nicht c), da $|1|=|-1|$)
\item $ xRy :\iff x\leq y$ erfüllt a), c), d) (nicht b), da $2\leq 3$ aber $2 \neq 3$
\end{enumerate}
Nun definieren wir die wichtigsten Arten von Relationen.

\subsection*{Definition 1.3.}
Eine Relation R auf X heißt
\begin{enumerate}[label=(\arabic*)]
\item \textbf{Äquivalenzrelation} falls sie reflexiv, symmetrisch und transitiv ist (typische Notation "~"),
\item \textbf{Ordnungsrelation} falls sie reflexiv, anti-symmetrisch und transitiv ist (Notation "$\leq$").
\end{enumerate}
Im Fall 2. heißt R auch \textbf{(partielle) Ordnung}, und falls 
$$ xRy \lor yRx \hspace*{2mm} \forall x,y \in X$$
zusätzlich gilt \textbf{totatle/ lineare Ordnung}.

\subsection*{Beispiel}
Kleiner Ausblick:
\begin{enumerate}[label=(\alph*)]
\item Zwei Matrizen $A,B \in \tn{Mat}_{n{\times}m}(K)$ heißen \textbf{ähnlich} falls $S \in \tn{GL}_n(K)$ existiert mit \\$B=S^{-1}{\cdot}A{\cdot}S$.
\item Sei M Menge und $X:=\mathfrak P(M)$ Potenzmenge von M\\
 Zu $A,B \in X$ definiere $A\leq B :\iff A\subseteq B$ eine Ordnungsrelation.
\end{enumerate}
%----------------------------------Ende Vorlesung 1----------------------------------
%Hier Vorlesung 2&3 rein!
%------------------------------------Vorlesung 4------------------------------------
\subsection*{Satz 2.6. (Lemma von Zorn)}
Sei $(X, \leq)$ geordnete Menge sodass jede Kette in $Y \subseteq X$ eine obere Schranke hat. Dann hat $X$ ein maximales Element.\\
\subsubsection*{Diskussion:}
\begin{enumerate}[label=(\arabic*)]
\item Nicht interessant (trivial) für $X$ endlich, oder für $(X,\leq )$ total geordnet: betrachte Kette $Y=X$.
\item Anwendungsfall oft $X\subseteq \mathfrak P(M)$. Kette $Y\subseteq X$ hat obere Schranke z.B. 
falls $\bigcup_{A\in Y}A\in X$ gilt.
\item Beweis ist technisch anspruchsvoller (Mengenlehre) und benutzt Auswahlaxiom; umgekehrt kann Auswahlaxium aus Zorn folgern (vgl. Halmos).
\end{enumerate}
Nun zu Basen für beliebige Vektorräume. \\
In Lineare Algebra I haben wir gesehen, dass jeder endlich erzeugte K-Vektorraum V eine (endliche) Basis $S\subseteq V$ hat; \\
Dann gilt $V\cong K^n$, $n:=\tn{dim}_KV$.\\
\underline{Erinnerung:} Eine Teilmenge $S\subseteq V$ heißt Erzeugendensystem falls 
$$span \hspace*{1mm} S := \{\sum_{i=1}^n
\lambda_i v_i | \lambda_i \in K, v_i \in S, n \in \mathbb N\} = V \tn{ gilt.}$$
Und $S\subseteq V$ heißt linear unabhängig falls $\sum_{i=1}^n \lambda_i v_i = 0$ mit $v_i\in S$ verschieden stets $\lambda_i = 0$, $\forall i$ impliziert.
\subsection*{Proposition 2.7.}
Zu I Menge sei $K^{(I)} := \{f:I\rightarrow K | f(i) \neq 0$ nur für endliche viele $i\in I\}$. \\
Dann ist $K^{(I)}$ ein K-Vektorraum mit Basis $\{e_i|i\in I\}$, wobei $e_i(j):= \twopartdef { 1 } {\tn{falls } j=i} {0} {\tn{sonst}}$. \\
Umgekehrt ist jeder K-Vektorraum mit Basis isomorph zu einem $K^{(I)}$.
\subsubsection*{Beweis:}
\begin{itemize}
\item[--] "K-Vektorraum": $K^{(I)}$ ist Untervektorraum von $K^I := \{f:I\rightarrow K\}$.
\item[--] "Erzeugendensystem": sei $f\in K^{(I)}$ und sei $I_0 := \{i_1,...,i_n\} \subseteq I$ mit $f(i) = 0$ $\forall i \notin I_0;$ setze $\lambda_k := f(i_k)$, $k=1,...n$ 
$$ \Rightarrow f = \sum_{k=1}^n \lambda_k e_{i_k}\tn{, denn }\lambda_j = f(i_j) = \sum_{k=1}^n \lambda_k e_{i_k}(i_j) = \lambda_j, \hspace*{1mm} \forall j = 1,...,n$$
\item[--] "linear unabhängig": angenommen $\sum \lambda_k e_{i_k} = 0 \Rightarrow \lambda_j = \sum \lambda_k e_{i_k}(i_j) = 0$ $\forall j$
\end{itemize}
Zusatz: Sei V ein K-Vektorraum mit Basis $\{e_i | i \in I\}$, so ist die Linearkombinationsabbildung (vgl. Lin-Alg.I, Bem. 8.3)
$\gamma: K^{(I)} \rightarrow V$, $f\mapsto \sum_{i\in I} f(i)e_i$ ein K-Isomorphismus.
\subsection*{Bemerkung}
$K^{(I)}=K^I \iff$ I endlich. \\
Aber z.B. ist $\mathbb Q^{(\mathbb N)}$ abzählbar und $\mathbb Q^\mathbb N$ überabzählbar (Übung). \\
Basen z.B. für $K^I$?
\subsection*{Satz 2.8.}
Jeder Vektorraum besitzt eine Basis.
\subsubsection*{Beweis (via Lemma von Zorn)}
Sei V ein K-Vektorraum und sei $X \subseteq \mathfrak P(V)$ die Menge aller linear unabhängigen Teilmengen $S \subseteq Y$. \\
Sei $Y\subseteq X$ Kette. Behauptung: $T:=\bigcup_{S\in Y} S$ ist linear unabhängig. \\
Gelte $\sum_{i=1}^n \lambda_i v_i = 0$ mit $\lambda_i \in K$ und $v_i \in T$ verschieden. \\
$$\Rightarrow \exists S_i \in Y\tn{ mit }v_i \in S_i\tn{, }\forall i=1,...n$$
$$\Rightarrow_{\tn(S_1,...,S_n \tn{Kette)}} \exists i_0\tn{ mit }S_i \subseteq S_{i_0}\tn{, } \forall i\tn{, somit }v_i \in S_{i_0}$$
$$\Rightarrow_{S_{i_0}\tn{ lin. unabh.}}\tn{ alle }\lambda_i=0$$
$$\Rightarrow_{\tn{Satz 2.6 (Zorn)}}\tn{ es existiert ein maximales Element S in X.}$$
Behauptung dieses S ist auch erzeugend (dann fertig).\\
Angenommen es existiert $v\in V\backslash\tn{span }S$ sei $S':= S\cup \{v\}$. Dann ist S' linear unabhängig.\\
Betrachte $\sum_{i=1}^n\lambda_iv_i=0$ mit $v_i\in S'$ verschieden.
\begin{itemize}
\item falls alle $v_i\in S \Rightarrow$ alle $\lambda_i = 0$
\item sonst $\lambda v = \sum \lambda_j v_j$, wäre $\lambda \neq 0 \Rightarrow v\in \tn{span }S \lightning$\\
$\Rightarrow \lambda = 0 \Rightarrow$ alle $\lambda_j=0$
\end{itemize}
Weil aber $S\subsetneq S'$ Widerspruch zur Maximalität von S.
%----------------------------------Ende Vorlesung 4----------------------------------
\end{document}