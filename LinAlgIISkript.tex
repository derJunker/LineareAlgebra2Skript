\documentclass {article}

\usepackage[a4paper, total={6in, 8in}]{geometry}

\usepackage{amssymb}
\usepackage{amsfonts}
\usepackage{enumitem}

\begin{document}
\title{Lineare Algebra II - Mitschrift}
\author{Sebastian Pretzsch, Finn Ribbeck, Jonas Heitz}
\maketitle

\section{Relationen}
Relationen beschreiben Beziehungen zwischen Elementen von Mengen. \\
Wdh. $X{\times}Y := \{(x,y) | x \in X, y \in Y\}$ \\
Menge der (geordneten) Paare, "kartesiches Produkt".

\subsection*{Definition 1.1.}
Seien X,Y Mengen. Eine \textbf{Relation} zwischen X und Y ist eine Teilmenge $R \subseteq X{\times}Y$. \\
\underline{Notation}: für $(x,y)\in R$ auch $xRy$. \\
Falls $X=Y$ sage auch Relation "auf X".

\subsection*{Beispiel}
$X$ Punktmenge, $Y$ Geradenmenge
$$xRy :\iff \textnormal{Punkt x liegt auf Gerade y}$$

\subsection*{Bemerkung}
Eine Abbildung (Funktion) $f:X{\to}Y$ weist zu jedem $x\in X$ genau ein $
y\in Y$ zu. \\
Wir werden diese fortan als spezielle Relation
$$R_f = \{(x, f(x)) | x\in X\} \subseteq X{\times}Y$$
auffassen. Wir betrachten insbesondere Relation auf X.

\subsection*{Definition 1.2.}
Sei X eine Menge. Eine Relation $R \subseteq X{\times}X$ ist
\begin{enumerate}[label=(\alph*)]
\item \textbf{reflexiv} falls $xRx \hspace*{2mm}\forall x\in X$,
\item \textbf{symmetrisch} falls $xRy \Rightarrow yRx \hspace*{2mm}\forall x,y\in X$ (denn auch "$\Leftarrow$" gilt),
\item \textbf{anti-symmetrisch} falls $xRy \land yRx \Rightarrow x=y$,
\item \textbf{transitiv} falls $xRy \land yRz \Rightarrow xRz$.
\end{enumerate}

\subsection*{Beispiel} 
Sei $X = \mathbb R$.
\begin{enumerate}[label=(\alph*)]
\item $ R:=\{(x,x) | x\in X\}$ (d.h. $xRy \Rightarrow x=y$) erfüllt a), b), c), d).
\item $ xRy :\iff |x|=|y|$ erfüllt a), b), d) (nicht c), da $|1|=|-1|$)
\item $ xRy :\iff x\leq y$ erfüllt a), c), d) (nicht b), da $2\leq 3$ aber $2 \neq 3$
\end{enumerate}
Nun definieren wir die wichtigsten Arten von Relationen.

\subsection*{Definition 1.3.}
Eine Relation R auf X heißt
\begin{enumerate}
\item \textbf{Äquivalenzrelation} falls sie reflexiv, symmetrisch und transitiv ist (typische Notation "~"),
\item \textbf{Ordnungsrelation} falls sie reflexiv, anti-symmetrisch und transitiv ist (Notation "$\leq$").
\end{enumerate}
Im Fall 2. heißt R auch \textbf{(partielle) Ordnung}, und falls 
$$ xRy \lor yRx \hspace*{2mm} \forall x,y \in X$$
zusätzlich gilt \textbf{totatle/ lineare Ordnung}.

\subsection*{Beispiel}
Kleiner Ausblick:
\begin{enumerate}[label=(\alph*)]
\item Zwei Matrizen $A,B \in \textnormal{Mat}_{n{\times}m}(K)$ heißen \textbf{ähnlich} falls $S \in \textnormal{GL}_n(K)$ existiert mit \\$B=S^{-1}{\cdot}A{\cdot}S$.
\item Sei M Menge und $X:=\mathfrak P(M)$ Potenzmenge von M\\
 Zu $A,B \in X$ definiere $A\leq B :\iff A\subseteq B$ eine Ordnungsrelation.
\end{enumerate}
\end{document}